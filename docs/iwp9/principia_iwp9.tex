\documentclass{article}

\title{
Principia Softwarica: Plan 9 explained
}
\author{
Yoann Padioleau\\
yoann.padioleau@gmail.com
}

\begin{document}
\maketitle

\begin{abstract}
An abstract
\end{abstract}

\section{Introduction}

% Plan9 didn't have the success it deserved. Sad.
% Elegant, small, designed by great minds. Not just kernel,
% they rethink the whole thing, including toolchain, windowing system, debuggers,
% network and graphics stack, build system.
% (cite plan9 classic paper, cite rio or predecessor).
%% sad it failed; Rob Pike then sad wrote "software research is irrelevant"

% Unfortunately Linux/GNU/Xfree86 won. Arguably not as innovative, not as elegant,
% not as simple, but it won. Being OSS and first.
% Plan9 was Open source too, but a bit late.
%%History repeat itself ... 
%%Unix v6 was closed by AT&T Freebsd in prison for a long time.
%% Plan9 was closed, then inferno, then open source but late, and
%% really open sourced far later by plan9 foundation.
% Linux/GNU/Xfree86 nice to have robust open source Unix, but hard to understand.
% Order of magnitude more code than Plan9.
% Compare LOC rc bs bash, LOC rio vs xterm. lol.

% Enter principia! Repurpose plan9 as a great teaching OS with OS in
% general sense, not just kernel but the whole thing!
% Xv6 (and Unix v6 before) great for kernel, but limited.  Principia
% will explain the whole thing!
% Plan9 actually incredibly small, given what it does! ex assembler is X LOC,
% linker is Y LOC.
% STEPS project, rethink full in far less LOC. goal was 10k, but failed,
% but plan9 is 300k, still really small!
%see-later: TABLE
% 10 000 LOC is explainable in a book of a reasonable size!
% How to explain it?
% Actually use technique, literate programming~cite{}, so that the whole code
% is in the book. Will give short intro to literate programming later.

% TOC:
% - Principia softwarica book set:
%   20 books! Each a literate system essential program for a developer!
%   different sections (repeat Principia.nw)
% - Literate programming intro
%   ex of book? or of ed.nw?
% - Yet another plan9 repo. 
%   Yet another fork. git. literate programs. Subset of plan9.
%   Diff with 9front, plan9port, etc. is focus on code explanations!
%   code reorg, more easily "discoverable" dirs (5a cryptic, it's
%   an assembler/).
% - Yet another distribution. Cross compiled from goken (see WIP), and 
%   make new distrib easily. Easy to experiment. Got trouble 10 years ago
%   to test, with ISO CDs, qemu pbs.
%   now qemu friendly, and raspberry pi friendly! great for teaching context!
%   cheap to buy. Greak work by Richard Miller (but again not easy to test
%   currently).
%   Incredible ease and speed! modify syscall order, recompile
%   everything, test GHA, it works!
% - Quality of life improvements:
%   * git, configure, docker, GHA
% - Current state and stats for the books

\section{Conclusion}

\end{document}
