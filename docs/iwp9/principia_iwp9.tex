\documentclass{article}

\usepackage{multirow}
\usepackage{array}

\title{
Principia Softwarica: Plan9 code explained
}
\author{
Yoann Padioleau\\
yoann.padioleau@gmail.com
}

\begin{document}
\maketitle

\begin{abstract}
An abstract
\end{abstract}

\section{Introduction}

Plan9 didn't have the success it deserved. 
% sad
It was
innovative,
elegant,
small,
and designed by great programmers.
%
Those great minds didn't just rethink the kernel, they rethinked
the whole operating system (OS) in a holistic manner with 
a kernel pushing the ``everything is a file'' motto to its limit,
an integrated graphic and network stack,
a windowing system that could even run under itself,
a compact cross-compiling toolchain 
(with assemblers, linkers, C compilers, debuggers, and profilers
for most architectures),
minimalist C libraries,
a simpler build system and
shell,
and more.
% just missed browser :( and being OSS from day one before 1991, and VCS


What is even more impressive is that those powerful programs
contain only in the order of thousands lines of code (LOC).
%
Rio for example, the Plan9 windowing system, contains just 10 000 LOC
and is arguably more poweful than XOrg (an implementation of XWindow)
which contains millions LOC. The Xwindow program xterm, which
is just a terminal, not the windowing system, contains already 80 000 LOC
while Rio contains a built-in terminal in less than 1000 LOC.
% that's a 80x reduction!
% (cite plan9 classic paper, cite rio or predecessor).
% possible because of elegance of plan9 namespace, /dev/cons, etc.
%
In fact, with only 350 000 LOC, Plan9 implemented from scratch all
the major components of an OS. 
%% sad it failed; Rob Pike then sad wrote "software research is irrelevant"


Instead, the Linux/GNU/Xorg OS won, partly because it
was the first open-source (OSS) system available.
% Arguably not as innovative, not as elegant, not as simple, but it won.
% Plan9 was Open source too, but a bit late.
%%History repeat itself ... 
%%Unix v6 was closed by AT&T Freebsd in prison for a long time.
%% Plan9 was closed, then inferno, then open source but late, and
%% really open sourced far later by plan9 foundation.
% Linux/GNU/Xfree86 nice to have robust open source Unix
Unfortuntately, even if the source code is open, it is very hard to understand
most of this code because each component (e.g., the Linux kernel,
the Bash shell, the GNU C library, gcc, binutils, gdb) contains 
multiple orders of magnitude more code than Plan9 with millions LOC.
% Compare LOC rc bs bash, LOC rio vs xterm. lol.
%see-later: TABLE
This situation is sad, and potentially dangerous, because most
programmers rely on a giant software stack that very few people (if any)
fully understand.
The situation is even worse for students who don't have any path
to understand deeply this software stack.

Enter {\em Principia Softwarica}, a new project to repurpose Plan9
from an old research OS to a fresh teaching OS,
with OS in a general sense: not just the kernel but the whole
software stack.

XXX
{\em Principia Softwarica} is a series of books, each explaining full
code of component for student;
accelerated path.
%alt: How to explain it?
Note that each book will contain the whole code!
definitive way to fully understand.
Literate programming, 
% Actually use technique, literate programming~cite{}, so that the whole code
% is in the book. Will give short intro to literate programming later.
% but first describe the books.
10 000 LOC is explainable in a book of a reasonable size!
bash for example? vs rc ?
millions LOC would require 400 books of 300 pages, just for bash :)

Understand essence, useful for many things.
Hopefully give the right place to show hidden gem that is Plan9 code!
and code of those great programmers.

The rest of the article is organized as follows:

% TOC:
% - Principia softwarica book set:
%   20 books! Each a literate system essential program for a developer!
% - Literate programming intro
% - Yet another plan9 repo. 
% - Yet another distribution
% - Modern software engineering practice
% - Future and related work.

\section{The bookset}

%trans:


\begin{tabular}{p{2.2cm} l p{2.5cm} r r r}
\hline
\textbf{Category} & \textbf{Book} & \textbf{Program(s)} & \textbf{LOC} & \textbf{LOE} & \textbf{Pages} \\
\hline

\multirow{3}{*}{Core system}
  & Kernel
  & \texttt{9pi}
  & 60 000 & 0 & 42  \\ \cline{2-6}

  & Core library
  & \texttt{libc libregexp libthread}
  & 21 500 & 0 & 42 \\ \cline{2-6}

  & Shell
  & \texttt{rc}
  & 7 500 & 0 & 42 \\

\hline

\multirow{3}{=}{Development\\ toolchain}
  & C compiler
  & \texttt{5c}
  & 25 000 & 0 & 42 \\ \cline{2-6}

  & Assembler
  & \texttt{5a}
  & 4 100 & 0 & 42 \\ \cline{2-6}

  & Linker
  & \texttt{5l}
  & 7 900 & 0 & 42 \\ \cline{2-6}

%skipped:
%  & Emulator
%  & \texttt{5i}
%  & 4 400 \\

%skipped: ocamlc, ocamlopt
%skipped: olex, oyacc

\hline

\multirow{4}{=}{Developer\\tools}
  & Editor
  & \texttt{ed}
  & 25 000 & 0 & 42 \\ \cline{2-6}
%skipped: editor efuns

  & Build system
  & \texttt{mk}
  & 25 000 & 0 & 42 \\ \cline{2-6}

  & Debugger
  & \texttt{db acid}
  & 4 100 & 0 & 42 \\ \cline{2-6}

  & Profiler
  & \texttt{prof}
  & 7 900 & 0 & 42 \\

%skipped: ogit
%skipped: troff

\hline

\multirow{2}{=}{Graphics}
  & Graphics stack
  & \texttt{/dev/draw lib\_graphics}
  & 26 000 & 0 & 42 \\ \cline{2-6}
% 10k+ (kernel) + 16k

  & Windowing system
  & \texttt{rio}
  & 10 500 & 0 & 42 \\

%skipped: libpanel

\hline

\multirow{1}{=}{Networking}
  & Network stack
  & \texttt{/dev/net lib\_networking}
  & 23 000 & 0 & 42 \\ %\cline{2-6}
% 23k+ (kernel) + 51k

%skipped: mmm

\hline

\multirow{1}{=}{Utilities}
  & CLI utilities
  & \texttt{cat ls grep sed diff tar gzip \ldots}
  & 23 000 & 0 & 42 \\ %\cline{2-6}
% also bc, hoc, awk, etc

\hline

\textbf{Total: } & & & 0 & 0 & 0 \\

\hline

% ARM (and raspberry pi) focus. simpler than 86, and goal is understand essence,
% not useless detail. But still real arch! not toy arch!
% ARM in most phones today and in raspberry pi!
% in fact kernel 9pi. Great work by Richard Miller not in original plan9.
% Raspberri pi great for teaching context.

%said-before? lots of pages, but if take Bash, would already be
% X pages!

% As can see, unfinished, still lots of work.
% I started in 2014 but was on pause between 2019 and 2024,
% and resume almost full-time on it mid 2025.

\end{tabular}


\section{Literate programming introduction}

%   noweb.
%   ex of book? or of ed.nw?
%   also syncweb! (also nice index, nice -lpizer)

\section{Yet another Plan9 fork}

% git URL. Yet another fork. 
%said-later: git is good

% literate programs. Subset of plan9.

%   Diff with 9front, plan9port, etc. is focus on code explanations!

%   code reorg, more easily "discoverable" dirs (5a cryptic, it's
%   an assembler/). Show output of tree command in 3 columns.

%   Actually simplified the code a bit sometimes, so easier to explain!
%   Some nice improvments IMHO. Found also bugs! as I was explaining
%   and could not explain :)


\section{A new Plan9 distribution}

% Cross compiled from goken (see WIP), and 
%   make new distrib easily. Easy to experiment. Got trouble 10 years ago
%   to test, with ISO CDs, qemu pbs. slow down adoption.
%   now qemu friendly, and raspberry pi friendly! great for teaching context!
%   cheap to buy. Greak work by Richard Miller (but again not easy to test
%   currently).
%   Incredible ease and speed! modify syscall order, recompile
%   everything, test GHA, it works! and can do that from Linux/macOS/Windows,
%   so easier to experiment and adopt! especially in teaching context

\section{Modern practice}
%alt: Quality of life improvements:
%alt: A modern distribution

% Modern does not always mean better, and very often old plan9
% was better than modern tools, still there are a few things
% that are convenient nowadays, even if arguably not most elegant:

% git and github! easy collaboration (9front too).
% docker build infra.
% GHA (just run docker build)

% C ref_xxx, error1, abuse int/string.
% less globals, better split files, less mutual deps (thx codegraph)

\section{Future work}
%said-before: current status in table section 2 now

% not self-hosted yet, build principia from other OS (which has
% its advantage in teaching context).

\section{Related work}

% Xv6 (and Unix v6 before) great for kernel, but limited.
% STEPS project, rethink full in far less LOC. goal was 10k, but failed,

\section{Conclusion}

\end{document}
