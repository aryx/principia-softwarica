%\section{About this document}

This document is a {\em literate program}~\cite{lp-book}. 
It derives from a set of files processed by a tool,
{\tt syncweb}~\cite{syncweb}, generating either
this book or the actual source code of the program. The code and
its documentation are thus strongly connected.
%old: ([[noweb]]~\cite{noweb} and [[syncweb]]~\cite{syncweb})
%pad: I now use mostly syncweb with -to_tex or -to_code,
% but I still use noweb.sty, so fair to keep the reference to noweb above?

%still? speak about FACEBOOK: special marks ?

%\t what is wrong with Javadoc or omcaldoc in our case ?
% I dont like it, cf sexp_int.mli, too many functions, given equal
% importance. Of course could reorder, but then that's the point, 
% you want to reorder => use LP. Moreover 2 views will always be better
% than just one view.



% speak more to explain LP for layman, with the chunks,
% possible multiple >>= on the same chunkname, etc.

% speak about type annotations? (// enum<...> )
%  paper: how infer those annotations!
% speak about context annotations (... -> <>), nice DSL :)
%  paper: how infer those annotations! context is super important to
%  help understand code (focus and context! classic)

% good side effect of LP, can more clearly show when things
% are independent, e.g. when aggregate in the "initialisaton" chunk,
% that means all initialisation are independent. If there is an order
% we would use another chunk and explicitly sequence them.

%paper: how auto generate those literate documents! Big project.
% see also plan-xix.org#Publish
